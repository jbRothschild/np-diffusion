

\begin{center}
\begin{tabular}{ |p{1.5cm}||p{14cm}|  }
 \hline
 \centering Year & Plan\\
 \hline
 \centering 1   & Conclude work on advection and diffusion through fenestrations, compare with experimental results.\\
                & Investigate different tranport models of nanoparticles through endothelial cells by endocytosis and determine corresponding distributions of the nanoparticles within tumor.\\
                & \st{Develop a model for protein adsorption using a spherical polymer brush formulation.}\\
 \centering 2   & \st{Probe the behaviour of the grafted polymer in different solutions of proteins using coarse-grained dynamical simulations.}\\
                & Explore models of diffusion and transport in tumor, preparing fluid dynamic simulations to test models. \\
 \centering 3   & \st{Evaluate kinetics of coarse-grained model interacting with other nanoparticles or membranes in protein serum.}\\
                & \st{Develop a population dynamics model for the composition of the corona.}\\
                & Incorporate further interactions into the model of the tumor microenvironment and change nature of nanoparticles in our simulations.\\
 \hline
\end{tabular}
\end{center}

Of the many aspects in tumor targeting that are unresolved, we aim to focus on the consequences of the corona formation on drug delivery.
First, we would construct some model of the forces that govern the adsorption of different proteins on the surface of the nanoparticles and study how the composition of the corona differs in different environments, varying interactions and initial conditions.
From this study, we can offer predictions as to how the composition of the corona can be manipulated by certain designs and properties of the nanoparticle surface.
Then we would study dynamics of these particles along with their adsorbed proteins, how they move in fluids and interact with components of the environemnt.
This will provide us with insight as to what factors influence the nature in serum of these particles.
Finally, we would like to investigate through simulations how these nanoparticles diffuse, constructing an interactive tumor environment and simulating the nanoparticles within.
We can explore the dominant potential barriers that hinder the circulation within the tumor and propose mechanisms that would overcome them.

This computational work aims to guide the design of nanoparticles that increases uptake and distribution in the tumor.
We intend to work closely with our experimental collaborators to facilitate the feedback on both ends.
Problems in the therapy strategies will surely be encountered however these can be further incorporated to our computational models and our hypothesis tested in a laboratory setting.
